% EDIT FROM HERE
% Example text
% Write your abstract here

The abstract is a very brief summary of the dissertation's contents. It should be about half a page long. 

\begin{itemize}
\item Audience: technical or academic peers.
\item Content: concise summary of the problem, methodology, results, and conclusions.
\item Language: uses discipline-specific terminology and assumes the reader has technical background knowledge.
\item Purpose: to give experts a quick overview of the work so they can decide whether to read the full document.
\end{itemize}

\textbf{Example of Abstract}\\
We propose a novel convex optimization framework for parameter identification in nonlinear dynamical systems using sparse measurements. The approach leverages structured regularization and is validated on a benchmark biochemical reaction model, showing improved accuracy and robustness compared to existing methods.


%%%%%%%%%%%%%%%%%%%%%%%%%%%%%%%%%%%%%%%
% DO NOT MODIFY FROM HERE ...
\cleardoublepage\phantomsection
\addcontentsline{toc}{chapter}{Plain Language Summary}\mtcaddchapter 
\chapter*{Plain Language Summary}
\addtocounter{counter}{-1}
% ... TO HERE
%%%%%%%%%%%%%%%%%%%%%%%%%%%%%%%%%%%%%%%

% EDIT FROM HERE
% Example text

Also this should be about half a page long.

\begin{itemize}
\item Audience: non-experts, such as policymakers, journalists, or the general public.
\item Content: explains why the research matters, what was done, and what the implications and impact are, without technical jargon.
\item Language: clear, simple, and accessible; often includes metaphors or examples to aid understanding. Do not use acronyms.
\item Purpose: to communicate the significance and relevance of the work to a broad audience.
\end{itemize}


\textbf{Example of Plain Language Summary (for the same project in the abstract! Can you see how different this is?)}\\
Understanding how complex systems like biological reactions behave is important in medicine and engineering. We developed a new mathematical technique that can help uncover how these systems work, even if we only have limited data. This could help researchers make better predictions and design more effective treatments.



